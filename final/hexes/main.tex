\documentclass[a4paper]{article}

%%% Поля и разметка страницы %%%
\usepackage{pdflscape} % Для включения альбомных страниц
\usepackage{geometry}  % Для последующего задания полей
\usepackage{fancyhdr}

%%% Выбор компилятора, кодировок для вёрстки (pdflatex, xelatex)
\usepackage{ifxetex}
\ifxetex
    \usepackage{polyglossia} % Поддержка многоязычности
\else
    \usepackage[T2A]{fontenc}
    \usepackage[utf8]{inputenc}
    % Поддержка русского языка (переносы, стилистика и т. д.)
    \usepackage[english, russian]{babel}
\fi

% Векторная графика
\usepackage{tikz}

%%% Изображения %%%
\usepackage{graphicx}

%%% Выравнивание и переносы %%%
% \sloppy % Избавляемся от переполнений
% \clubpenalty=10000 % Запрещаем разрыв страницы после первой строки абзаца
% \widowpenalty=10000 % Запрещаем разрыв страницы после последней строки абзаца

\setlength{\parindent}{0pt} % Абзацный отступ

% Значения отступов полей
\geometry{a4paper,%
  top = 10mm,%
  bottom = 15mm,%
  left = 10mm,%
  right = 10mm%
}

%%% Кодировки и шрифты %%%
\ifxetex{}
    \setmainlanguage[babelshorthands=true]{russian}  % Язык по-умолчанию русский
    \setotherlanguage{english}                       % Дополнительный язык
                                                     % английский

    \defaultfontfeatures{Ligatures=TeX,Mapping=tex-text}
    \setmainfont{Times New Roman}
    \newfontfamily\cyrillicfont{Times New Roman}

    \setsansfont{Arial}
    \newfontfamily\cyrillicfontsf{Arial}

    \setmonofont{TerminusTTF}
    \newfontfamily\cyrillicfonttt{TerminusTTF}
\fi

\renewcommand{\headrulewidth}{0pt}

\newcommand{\fillshape}[3]{%
  \begin{scope}
    \clip #1;
    \node {\includegraphics[#3]{#2}};
  \end{scope}
  \draw[line width=0.0mm] #1;
}

\newcommand{\fillhexagon}[2][]{%
  \fillshape{(0:2.81) -- (60:2.81) -- (120:2.81) -- (180:2.81) -- (240:2.81) -- (300:2.81) --
    cycle}{#2}{#1}%
}

\newcommand{\picturePage}[9]{%
  \begin{tikzpicture}
    \begin{scope}
      #1%
    \end{scope}
    \begin{scope}[xshift=42.3mm,yshift=-24.3mm]%
      #2%
    \end{scope}
    \begin{scope}[xshift=84.6mm]%
      #3%
    \end{scope}
    \begin{scope}[xshift=126.9mm,yshift=-24.3mm]%
      #4%
    \end{scope}

    \begin{scope}[yshift=-48.8mm]
      #5%
    \end{scope}
    \begin{scope}[xshift=42.3mm,yshift=-73.1mm]%
      #6%
    \end{scope}
    \begin{scope}[xshift=84.6mm,yshift=-48.8mm]%
      #7%
    \end{scope}
    \begin{scope}[xshift=126.9mm,yshift=-73.1mm]%
      #8%
    \end{scope}

    \begin{scope}[yshift=2*-48.8mm]
      #9%
    \end{scope}

    \picturePageContFirst%
}

\newcommand{\picturePageContFirst}[9]{%
    \begin{scope}[xshift=42.3mm,yshift=2*-48.8mm-24.3mm]%
      #1%
    \end{scope}
    \begin{scope}[xshift=84.6mm,yshift=2*-48.8mm]%
      #2%
    \end{scope}
    \begin{scope}[xshift=126.9mm,yshift=2*-48.8mm-24.3mm]%
      #3%
    \end{scope}

    \begin{scope}[yshift=3*-48.8mm]
      #4%
    \end{scope}
    \begin{scope}[xshift=42.3mm,yshift=3*-48.8mm-24.3mm]%
      #5%
    \end{scope}
    \begin{scope}[xshift=84.6mm,yshift=3*-48.8mm]%
      #6%
    \end{scope}
    \begin{scope}[xshift=126.9mm,yshift=3*-48.8mm-24.3mm]%
      #7%
    \end{scope}

    \begin{scope}[yshift=4*-48.8mm]
      #8%
    \end{scope}
    \begin{scope}[xshift=42.3mm,yshift=4*-48.8mm-24.3mm]%
      #9%
    \end{scope}

    \picturePageContSecond%
}

\newcommand{\picturePageContSecond}[2]{%
    \begin{scope}[xshift=84.6mm,yshift=4*-48.8mm]%
      #1%
    \end{scope}
    \begin{scope}[xshift=126.9mm,yshift=4*-48.8mm-24.3mm]%
      #2%
    \end{scope}

  \end{tikzpicture}
}

\newcommand{\picturePageReverse}[9]{%
  \begin{tikzpicture}
    \begin{scope}[yshift=-24.3mm]%
      #1%
    \end{scope}
    \begin{scope}[xshift=42.3mm]%
      #2%
    \end{scope}
    \begin{scope}[xshift=84.6mm,yshift=-24.3mm]%
      #3%
    \end{scope}
    \begin{scope}[xshift=126.9mm]%
      #4%
    \end{scope}

    \begin{scope}[yshift=-48.8mm-24.3mm]
      #5%
    \end{scope}
    \begin{scope}[xshift=42.3mm,yshift=-48.8mm]%
      #6%
    \end{scope}
    \begin{scope}[xshift=84.6mm,yshift=-48.8mm-24.3mm]%
      #7%
    \end{scope}
    \begin{scope}[xshift=126.9mm,yshift=-48.8mm]%
      #8%
    \end{scope}

    \begin{scope}[yshift=2*-48.8mm-24.3mm]
      #9%
    \end{scope}

    \picturePageReverseContFirst%
}

\newcommand{\picturePageReverseContFirst}[9]{%
    \begin{scope}[xshift=42.3mm,yshift=2*-48.8mm]%
      #1%
    \end{scope}
    \begin{scope}[xshift=84.6mm,yshift=2*-48.8mm-24.3mm]%
      #2%
    \end{scope}
    \begin{scope}[xshift=126.9mm,yshift=2*-48.8mm]%
      #3%
    \end{scope}

    \begin{scope}[yshift=3*-48.8mm-24.3mm]
      #4%
    \end{scope}
    \begin{scope}[xshift=42.3mm,yshift=3*-48.8mm]%
      #5%
    \end{scope}
    \begin{scope}[xshift=84.6mm,yshift=3*-48.8mm-24.3mm]%
      #6%
    \end{scope}
    \begin{scope}[xshift=126.9mm,yshift=3*-48.8mm]%
      #7%
    \end{scope}

    \begin{scope}[yshift=4*-48.8mm-24.3mm]
      #8%
    \end{scope}
    \begin{scope}[xshift=42.3mm,yshift=4*-48.8mm]%
      #9%
    \end{scope}

    \picturePageReverseContSecond%
}

\newcommand{\picturePageReverseContSecond}[2]{%
    \begin{scope}[xshift=84.6mm,yshift=4*-48.8mm-24.3mm]%
      #1%
    \end{scope}
    \begin{scope}[xshift=126.9mm,yshift=4*-48.8mm]%
      #2%
    \end{scope}

  \end{tikzpicture}
}

\newcommand{\asteroids}{%
  \fillhexagon[]{../../production/hexes/deep_space_tiles/asteroids.png}
}

\newcommand{\nebula}{%
  \fillhexagon[]{../../production/hexes/deep_space_tiles/nebula.png}
}

\newcommand{\supernova}{%
  \fillhexagon[]{../../production/hexes/deep_space_tiles/supernova.png}
}

\newcommand{\minerals}{%
  \fillhexagon[]{../../production/hexes/deep_space_tiles/minerals.png}
}

\newcommand{\lostInSpace}{%
  \fillhexagon[]{../../production/hexes/deep_space_tiles/lost_in_space.png}
}

\newcommand{\doomsdayMachineFirst}{%
  \fillhexagon[]{../../production/hexes/deep_space_tiles/doomsday_machine_1.png}
}

\newcommand{\doomsdayMachineSecond}{%
  \fillhexagon[]{../../production/hexes/deep_space_tiles/doomsday_machine_2.png}
}

\newcommand{\doomsdayMachineThird}{%
  \fillhexagon[]{../../production/hexes/deep_space_tiles/doomsday_machine_3.png}
}

\newcommand{\danger}{%
  \fillhexagon[]{../../production/hexes/deep_space_tiles/danger.png}
}

\newcommand{\blackHole}{%
  \fillhexagon[]{../../production/hexes/deep_space_tiles/black_hole.png}
}

\newcommand{\warpPointFirst}{%
  \fillhexagon[]{../../production/hexes/deep_space_tiles/warp_point_1.png}
}

\newcommand{\warpPointSecond}{%
  \fillhexagon[]{../../production/hexes/deep_space_tiles/warp_point_2.png}
}

\newcommand{\spaceWreck}{%
  \fillhexagon[]{../../production/hexes/deep_space_tiles/space_wreck.png}
}

\newcommand{\abydos}{%
  \fillhexagon[]{../../production/hexes/deep_space_tiles/planets/abydos.png}
}
\newcommand{\ada}{%
  \fillhexagon[]{../../production/hexes/deep_space_tiles/planets/ada.png}
}
\newcommand{\aries}{%
  \fillhexagon[]{../../production/hexes/deep_space_tiles/planets/aries.png}
}
\newcommand{\arrakis}{%
  \fillhexagon[]{../../production/hexes/deep_space_tiles/planets/arrakis.png}
}
\newcommand{\babbage}{%
  \fillhexagon[]{../../production/hexes/deep_space_tiles/planets/babbage.png}
}
\newcommand{\centauri}{%
  \fillhexagon[]{../../production/hexes/deep_space_tiles/planets/centauri.png}
}
\newcommand{\cobol}{%
  \fillhexagon[]{../../production/hexes/deep_space_tiles/planets/cobol.png}
}
\newcommand{\cygni}{%
  \fillhexagon[]{../../production/hexes/deep_space_tiles/planets/cygni.png}
}
\newcommand{\deneb}{%
  \fillhexagon[]{../../production/hexes/deep_space_tiles/planets/deneb.png}
}
\newcommand{\gath}{%
  \fillhexagon[]{../../production/hexes/deep_space_tiles/planets/gath.png}
}
\newcommand{\romulus}{%
  \fillhexagon[]{../../production/hexes/deep_space_tiles/planets/romulus.png}
}
\newcommand{\rukbat}{%
  \fillhexagon[]{../../production/hexes/deep_space_tiles/planets/rukbat.png}
}

\newcommand{\reverseStub}{%
  \fillhexagon[]{../../production/hexes/reverse/border_reverse_gray.png}
}
\newcommand{\reverseGray}{%
  \fillhexagon[]{../../production/hexes/reverse/reverse_gray.png}
}
\newcommand{\reverseGreen}{%
  \fillhexagon[]{../../production/hexes/reverse/reverse_green.png}
}
\newcommand{\reverseBlue}{%
  \fillhexagon[]{../../production/hexes/reverse/reverse_blue.png}
}
\newcommand{\reverseRed}{%
  \fillhexagon[]{../../production/hexes/reverse/reverse_red.png}
}
\newcommand{\reverseYellow}{%
  \fillhexagon[]{../../production/hexes/reverse/reverse_yellow.png}
}
\newcommand{\altairRed}{%
  \fillhexagon[]{../../production/hexes/home_system_tiles/homeworlds/altair_red.png}
}
\newcommand{\altairRedReverse}{%
  \fillhexagon[]{../../production/hexes/home_system_tiles/homeworlds/altair_red_reverse.png}
}
\newcommand{\chulakGreen}{%
  \fillhexagon[]{../../production/hexes/home_system_tiles/homeworlds/chulak_green.png}
}
\newcommand{\chulakGreenReverse}{%
  \fillhexagon[]{../../production/hexes/home_system_tiles/homeworlds/chulak_green_reverse.png}
}
\newcommand{\terraBlue}{%
  \fillhexagon[]{../../production/hexes/home_system_tiles/homeworlds/terra_blue.png}
}
\newcommand{\terraBlueReverse}{%
  \fillhexagon[]{../../production/hexes/home_system_tiles/homeworlds/terra_blue_reverse.png}
}
\newcommand{\vasyrYellow}{%
  \fillhexagon[]{../../production/hexes/home_system_tiles/homeworlds/vasyr_yellow.png}
}
\newcommand{\vasyrYellowReverse}{%
  \fillhexagon[]{../../production/hexes/home_system_tiles/homeworlds/vasyr_yellow_reverse.png}
}

\newcommand{\nebulaBlue}{%
  \fillhexagon[]{../../production/hexes/home_system_tiles/nebula/nebula_blue.png}
}
\newcommand{\nebulaGreen}{%
  \fillhexagon[]{../../production/hexes/home_system_tiles/nebula/nebula_green.png}
}
\newcommand{\nebulaRed}{%
  \fillhexagon[]{../../production/hexes/home_system_tiles/nebula/nebula_red.png}
}
\newcommand{\nebulaYellow}{%
  \fillhexagon[]{../../production/hexes/home_system_tiles/nebula/nebula_yellow.png}
}

\newcommand{\asteroidsBlue}{%
  \fillhexagon[]{../../production/hexes/home_system_tiles/asteroids/asteroids_blue.png}
}
\newcommand{\asteroidsGreen}{%
  \fillhexagon[]{../../production/hexes/home_system_tiles/asteroids/asteroids_green.png}
}
\newcommand{\asteroidsRed}{%
  \fillhexagon[]{../../production/hexes/home_system_tiles/asteroids/asteroids_red.png}
}
\newcommand{\asteroidsYellow}{%
  \fillhexagon[]{../../production/hexes/home_system_tiles/asteroids/asteroids_yellow.png}
}

\newcommand{\mineralsBlue}{%
  \fillhexagon[]{../../production/hexes/home_system_tiles/minerals/minerals_5_blue.png}
}
\newcommand{\mineralsGreen}{%
  \fillhexagon[]{../../production/hexes/home_system_tiles/minerals/minerals_5_green.png}
}
\newcommand{\mineralsRed}{%
  \fillhexagon[]{../../production/hexes/home_system_tiles/minerals/minerals_5_red.png}
}
\newcommand{\mineralsYellow}{%
  \fillhexagon[]{../../production/hexes/home_system_tiles/minerals/minerals_5_yellow.png}
}

\newcommand{\blackHoleBlue}{%
  \fillhexagon[]{../../production/hexes/home_system_tiles/black_hole/black_hole_blue.png}
}
\newcommand{\blackHoleGreen}{%
  \fillhexagon[]{../../production/hexes/home_system_tiles/black_hole/black_hole_green.png}
}
\newcommand{\blackHoleRed}{%
  \fillhexagon[]{../../production/hexes/home_system_tiles/black_hole/black_hole_red.png}
}
\newcommand{\blackHoleYellow}{%
  \fillhexagon[]{../../production/hexes/home_system_tiles/black_hole/black_hole_yellow.png}
}

\newcommand{\andromeda}{%
  \fillhexagon[]{../../production/hexes/home_system_tiles/planets/blue/andromeda.png}
}
\newcommand{\barrenBlue}{%
  \fillhexagon[]{../../production/hexes/home_system_tiles/planets/blue/barren.png}
}
\newcommand{\bethel}{%
  \fillhexagon[]{../../production/hexes/home_system_tiles/planets/blue/bethel.png}
}
\newcommand{\eden}{%
  \fillhexagon[]{../../production/hexes/home_system_tiles/planets/blue/eden.png}
}
\newcommand{\odyssey}{%
  \fillhexagon[]{../../production/hexes/home_system_tiles/planets/blue/odyssey.png}
}
\newcommand{\orion}{%
  \fillhexagon[]{../../production/hexes/home_system_tiles/planets/blue/orion.png}
}
\newcommand{\prometheus}{%
  \fillhexagon[]{../../production/hexes/home_system_tiles/planets/blue/prometheus.png}
}
\newcommand{\rigel}{%
  \fillhexagon[]{../../production/hexes/home_system_tiles/planets/blue/rigel.png}
}
\newcommand{\vulcan}{%
  \fillhexagon[]{../../production/hexes/home_system_tiles/planets/blue/vulcan.png}
}

\newcommand{\bajor}{%
  \fillhexagon[]{../../production/hexes/home_system_tiles/planets/green/bajor.png}
}
\newcommand{\barrenGreen}{%
  \fillhexagon[]{../../production/hexes/home_system_tiles/planets/green/barren.png}
}
\newcommand{\castor}{%
  \fillhexagon[]{../../production/hexes/home_system_tiles/planets/green/castor.png}
}
\newcommand{\dakara}{%
  \fillhexagon[]{../../production/hexes/home_system_tiles/planets/green/dakara.png}
}
\newcommand{\eccles}{%
  \fillhexagon[]{../../production/hexes/home_system_tiles/planets/green/eccles.png}
}
\newcommand{\kronos}{%
  \fillhexagon[]{../../production/hexes/home_system_tiles/planets/green/kronos.png}
}
\newcommand{\pleiades}{%
  \fillhexagon[]{../../production/hexes/home_system_tiles/planets/green/pleiades.png}
}
\newcommand{\pollux}{%
  \fillhexagon[]{../../production/hexes/home_system_tiles/planets/green/pollux.png}
}
\newcommand{\sirius}{%
  \fillhexagon[]{../../production/hexes/home_system_tiles/planets/green/sirius.png}
}

\newcommand{\anthares}{%
  \fillhexagon[]{../../production/hexes/home_system_tiles/planets/red/anthares.png}
}
\newcommand{\athos}{%
  \fillhexagon[]{../../production/hexes/home_system_tiles/planets/red/athos.png}
}
\newcommand{\barrenRed}{%
  \fillhexagon[]{../../production/hexes/home_system_tiles/planets/red/barren.png}
}
\newcommand{\cerberus}{%
  \fillhexagon[]{../../production/hexes/home_system_tiles/planets/red/cerberus.png}
}
\newcommand{\essen}{%
  \fillhexagon[]{../../production/hexes/home_system_tiles/planets/red/essen.png}
}
\newcommand{\fionn}{%
  \fillhexagon[]{../../production/hexes/home_system_tiles/planets/red/fionn.png}
}
\newcommand{\omicron}{%
  \fillhexagon[]{../../production/hexes/home_system_tiles/planets/red/omicron.png}
}
\newcommand{\sheldon}{%
  \fillhexagon[]{../../production/hexes/home_system_tiles/planets/red/sheldon.png}
}
\newcommand{\vortigern}{%
  \fillhexagon[]{../../production/hexes/home_system_tiles/planets/red/vortigern.png}
}

\newcommand{\arcturus}{%
  \fillhexagon[]{../../production/hexes/home_system_tiles/planets/yellow/arcturus.png}
}
\newcommand{\aslak}{%
  \fillhexagon[]{../../production/hexes/home_system_tiles/planets/yellow/aslak.png}
}
\newcommand{\barrenYellow}{%
  \fillhexagon[]{../../production/hexes/home_system_tiles/planets/yellow/barren.png}
}
\newcommand{\benden}{%
  \fillhexagon[]{../../production/hexes/home_system_tiles/planets/yellow/benden.png}
}
\newcommand{\haldir}{%
  \fillhexagon[]{../../production/hexes/home_system_tiles/planets/yellow/haldir.png}
}
\newcommand{\tempe}{%
  \fillhexagon[]{../../production/hexes/home_system_tiles/planets/yellow/tempe.png}
}
\newcommand{\valhalla}{%
  \fillhexagon[]{../../production/hexes/home_system_tiles/planets/yellow/valhalla.png}
}
\newcommand{\vega}{%
  \fillhexagon[]{../../production/hexes/home_system_tiles/planets/yellow/vega.png}
}
\newcommand{\xiYellow}{%
  \fillhexagon[]{../../production/hexes/home_system_tiles/planets/yellow/xi.png}
}

\begin{document}

\thispagestyle{fancy}
\fancyhf{}
\cfoot{Deep Space System Hexes 1 of 5 (Front Side)}

\picturePage%
{\asteroids }%
{\asteroids }%
{\asteroids }%
{\asteroids }%
{\asteroids }%
{\asteroids }%
{\asteroids }%
{\asteroids }%
{\asteroids }%
{\asteroids }%
{\nebula    }%
{\nebula    }%
{\nebula    }%
{\nebula    }%
{\nebula    }%
{\nebula    }%
{\nebula    }%
{\nebula    }%
{\nebula    }%
{\nebula    }%

\newpage%

\thispagestyle{fancy}
\fancyhf{}
\cfoot{Deep Space System Hexes 2 of 5 (Front Side)}

\picturePage%
{\supernova   }%
{\supernova   }%
{\supernova   }%
{\minerals    }%
{\minerals    }%
{\minerals    }%
{\minerals    }%
{\minerals    }%
{\minerals    }%
{\minerals    }%
{\minerals    }%
{\minerals    }%
{\minerals    }%
{\minerals    }%
{\minerals    }%
{\minerals    }%
{\minerals    }%
{\lostInSpace }%
{\lostInSpace }%
{\lostInSpace }%

\newpage%

\thispagestyle{fancy}
\fancyhf{}
\cfoot{Deep Space System Hexes 3 of 5 (Front Side)}

\picturePage%
{\lostInSpace           }%
{\lostInSpace           }%
{\doomsdayMachineFirst  }%
{\doomsdayMachineSecond }%
{\doomsdayMachineThird  }%
{\danger                }%
{\danger                }%
{\danger                }%
{\danger                }%
{\danger                }%
{\danger                }%
{\danger                }%
{\danger                }%
{\danger                }%
{\danger                }%
{\danger                }%
{\danger                }%
{\danger                }%
{\danger                }%
{\danger                }%

\newpage%

\thispagestyle{fancy}
\fancyhf{}
\cfoot{Deep Space System Hexes 4 of 5 (Front Side)}

\picturePage%
{\danger    }%
{\danger    }%
{\danger    }%
{\danger    }%
{\danger    }%
{\danger    }%
{\danger    }%
{\danger    }%
{\danger    }%
{\danger    }%
{\danger    }%
{\blackHole }%
{\blackHole }%
{\blackHole }%
{\blackHole }%
{\blackHole }%
{\blackHole }%
{\blackHole }%
{\blackHole }%
{\blackHole }%

\newpage%

\thispagestyle{fancy}
\fancyhf{}
\cfoot{Deep Space System Hexes 5 of 5 (Front Side)}

\picturePage%
{\blackHole       }%
{\warpPointFirst  }%
{\warpPointFirst  }%
{\warpPointFirst  }%
{\warpPointSecond }%
{\warpPointSecond }%
{\warpPointSecond }%
{\spaceWreck      }%
{\spaceWreck      }%
{\spaceWreck      }%
{\abydos          }%
{\ada             }%
{\aries           }%
{\arrakis         }%
{\babbage         }%
{\centauri        }%
{\cobol           }%
{\cygni           }%
{\deneb           }%
{\gath            }%

\newpage%

\newgeometry{top = 10mm,%
  bottom = 15mm,%
  left = 16.87mm,%
  right = 10mm%
}

\thispagestyle{fancy}
\fancyhf{}
\cfoot{Deep Space System Hexes 1 of 1 (Back Side)}

\picturePageReverse%
{\reverseGray}%
{\reverseGray}%
{\reverseGray}%
{\reverseGray}%
{\reverseGray}%
{\reverseGray}%
{\reverseGray}%
{\reverseGray}%
{\reverseGray}%
{\reverseGray}%
{\reverseGray}%
{\reverseGray}%
{\reverseGray}%
{\reverseGray}%
{\reverseGray}%
{\reverseGray}%
{\reverseGray}%
{\reverseGray}%
{\reverseGray}%
{\reverseGray}%

\restoregeometry%

\newpage%

\thispagestyle{fancy}
\fancyhf{}
\cfoot{Home System Hexes 1 of 5 (Front Side)}

\picturePage%
{\nebulaBlue      }%
{\nebulaGreen     }%
{\nebulaRed       }%
{\nebulaYellow    }%
{\nebulaBlue      }%
{\nebulaGreen     }%
{\nebulaRed       }%
{\nebulaYellow    }%
{\asteroidsBlue   }%
{\asteroidsGreen  }%
{\asteroidsRed    }%
{\asteroidsYellow }%
{\asteroidsBlue   }%
{\asteroidsGreen  }%
{\asteroidsRed    }%
{\asteroidsYellow }%
{\mineralsBlue}%
{\mineralsGreen}%
{\mineralsRed}%
{\mineralsYellow}%

\newpage%

\thispagestyle{fancy}
\fancyhf{}
\cfoot{Home System Hexes 2 of 5 (Front Side)}

\picturePage%
{\mineralsBlue}%
{\mineralsGreen}%
{\mineralsRed}%
{\mineralsYellow}%
{\mineralsBlue}%
{\mineralsGreen}%
{\mineralsRed}%
{\mineralsYellow}%
{\mineralsBlue}%
{\mineralsGreen}%
{\mineralsRed}%
{\mineralsYellow}%
{\mineralsBlue}%
{\mineralsGreen}%
{\mineralsRed}%
{\mineralsYellow}%
{\mineralsBlue}%
{\mineralsGreen}%
{\mineralsRed}%
{\mineralsYellow}%

\newpage%

\thispagestyle{fancy}
\fancyhf{}
\cfoot{Home System Hexes 3 of 5 (Front Side)}

\picturePage%
{\mineralsBlue}%
{\mineralsGreen}%
{\mineralsRed}%
{\mineralsYellow}%
{\mineralsBlue}%
{\mineralsGreen}%
{\mineralsRed}%
{\mineralsYellow}%
{\mineralsBlue}%
{\mineralsGreen}%
{\mineralsRed}%
{\mineralsYellow}%
{\mineralsBlue}%
{\mineralsGreen}%
{\mineralsRed}%
{\mineralsYellow}%
{\mineralsBlue}%
{\mineralsGreen}%
{\mineralsRed}%
{\mineralsYellow}%

\newpage%

\thispagestyle{fancy}
\fancyhf{}
\cfoot{Home System Hexes 4 of 5 (Front Side)}

\picturePage%
{\blackHoleBlue}%
{\blackHoleGreen}%
{\blackHoleRed}%
{\blackHoleYellow}%
{\andromeda    }%
{\bajor        }%
{\anthares     }%
{\arcturus     }%
{\barrenBlue   }%
{\barrenGreen  }%
{\athos        }%
{\aslak        }%
{\bethel       }%
{\castor       }%
{\barrenRed    }%
{\barrenYellow }%
{\eden         }%
{\dakara       }%
{\cerberus     }%
{\benden       }%

\newpage%

\thispagestyle{fancy}
\fancyhf{}
\cfoot{Home System Hexes 5 of 5 (Front Side)}

\picturePage%
{\odyssey      }%
{\eccles       }%
{\essen        }%
{\haldir       }%
{\orion        }%
{\kronos       }%
{\fionn        }%
{\tempe        }%
{\prometheus   }%
{\pleiades     }%
{\omicron      }%
{\valhalla     }%
{\rigel        }%
{\pollux       }%
{\sheldon      }%
{\vega         }%
{\vulcan       }%
{\sirius       }%
{\vortigern    }%
{\xiYellow     }%

\newpage%

\newgeometry{top = 10mm,%
  bottom = 15mm,%
  left = 16.87mm,%
  right = 10mm%
}

\thispagestyle{fancy}
\fancyhf{}
\cfoot{Home System Hexes 1 of 1 (Back Side)}

\picturePageReverse%
{\reverseYellow      }%
{\reverseRed         }%
{\reverseGreen       }%
{\reverseBlue        }%
{\reverseYellow      }%
{\reverseRed         }%
{\reverseGreen       }%
{\reverseBlue        }%
{\reverseYellow      }%
{\reverseRed         }%
{\reverseGreen       }%
{\reverseBlue        }%
{\reverseYellow      }%
{\reverseRed         }%
{\reverseGreen       }%
{\reverseBlue        }%
{\reverseYellow      }%
{\reverseRed         }%
{\reverseGreen       }%
{\reverseBlue        }%

\restoregeometry%

\newpage%

\thispagestyle{fancy}
\fancyhf{}
\cfoot{Mix Hexes 1 of 1 (Front Side)}

\picturePage%
{\romulus}%
{\rukbat}%
{\terraBlue       }%
{\chulakGreen     }%
{\altairRed       }%
{\vasyrYellow     }%
{\reverseStub}%
{\reverseStub}%
{\reverseStub}%
{\reverseStub}%
{\reverseStub}%
{\reverseStub}%
{\reverseStub}%
{\reverseStub}%
{\reverseStub}%
{\reverseStub}%
{\reverseStub}%
{\reverseStub}%
{\reverseStub}%
{\reverseStub}%


\newpage%

\newgeometry{top = 10mm,%
  bottom = 15mm,%
  left = 16.87mm,%
  right = 10mm%
}

\thispagestyle{fancy}
\fancyhf{}
\cfoot{Mix Hexes 1 of 1 (Back Side)}

\picturePageReverse%
{\chulakGreenReverse }%
{\terraBlueReverse   }%
{\reverseGray}%
{\reverseGray}%
{\reverseStub}%
{\reverseStub}%
{\vasyrYellowReverse }%
{\altairRedReverse   }%
{\reverseStub}%
{\reverseStub}%
{\reverseStub}%
{\reverseStub}%
{\reverseStub}%
{\reverseStub}%
{\reverseStub}%
{\reverseStub}%
{\reverseStub}%
{\reverseStub}%
{\reverseStub}%
{\reverseStub}%

\restoregeometry%

\end{document}
